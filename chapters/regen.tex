\pagebreak
\section{Regeneration}\label{chap:regen}
\subsection{Regenerating random walker}\label{chap:regen-walker}
When regeneration is considered, the random walk that was used to describe a fight in Chapter~\ref{chap:fightDef} gets new edges. Now it will be possible to walk backwards as well as forwards in the state space. Unfortunately, since regeneration attempts happen with a fixed period $T_R$, the transition probabilities will vary with time. Furthermore, when the regeneration attempts do occur they might not coincide with the damaging events.

To incorporate the two events in the same random walker formalism, consider the time interval $\Delta t_k = [t_k, t_{k+1}-1]$, where $t_k$ is the (game)tick on which the $k$th hit occurs. A single transition of the random walker is now determined by the overall state change that takes place during time interval $\Delta t_k$. Since we have assumed $T_R \geq T_A$ the number of regeneration attempts during $\Delta t_k$ is at most 1.
The assumption $T_A \leq T_B$ should hold for nearly all cases in practice as the the typical regeneration period is 60 seconds and even the slowest weapons have attack periods of just 4.2 seconds. For some rarer cases such as flinching an enemy with unusually high regeneration rate where this could become a problem one could simply allow the regeneration of more than 1 hitpoint per attack interval.

The only source of randomness in the regeneration model is the tick $\tau$ on which the first regeneration attempt occurs. We treat it as a random variable distributed uniformly in the interval $[1, T_R]$. Alternatively $\tau$ can be thought of as the time until the next regeneration attempt at the beginning of a fight. Now the expected length of a fight is
\begin{align}
	\langle L_j \rangle
		&= \sum_{n=1}^{\infty}n\Pr{L_j=\,n}\nonumber\\
		&= \sum_{n=1}^{\infty}n\sum_{\tau=1}^{T_R}\Pr{\tau}\Pr{L_j=\,n \mid \tau}\nonumber\\
		&= \frac{1}{T_R}\sum_{\tau=1}^{T_R}\sum_{n=1}^{\infty}n\Pr{L_j=\,n \mid \tau}\nonumber\\
		&= \frac{1}{T_R}\sum_{\tau=1}^{T_R} \langle L_j^\tau \rangle
\end{align}
where $L_j^\tau$ is the length of a fight at the beginning of which the next regeneration attempt is $\tau$ ticks away. While not shown here, by almost identical reasoning to that in Chapter~\ref{chap:fightDef} one can derive the recurrence relation
\begin{align}\label{eq:regenrecurrence}
	\langle L_j^\tau \rangle
		&= 1 + \sum_{i=1}^{h} p_{ij}^\tau \langle L_i^{\tau - T_A} \rangle.
\end{align}
The only critical difference besides the transition probabilities is the recursion argument. Now in addition to modifying the remaining hitpoints, hitting the enemy also shifts $\tau$ backwards by $T_A$ ticks since this is the amount of time that passes between hits. The time-dependent transition probabilities are given by
\begin{align}
    p_{ij}^\tau
        &= \begin{cases}
			p_{ij} \quad &\mbox{if } \tau > T_A \pmod {T_R} \\
			p_{i,\min(j+1,h)} \quad &\mbox{if } \tau \leq T_A \pmod {T_R}
		\end{cases}\label{eq:damageDistribution}.
\end{align}
where $p_{ij}$ is the transition probability of the non-regenerating case (eq\ref{eq:noregenProb}). The minimum is taken to prevent hitpoints from exceeding $h$ and modular arithmetic used to handle the periodicity of the regeneration cycle.

Because of the backwards transition that regeneration has made possible, all states are now dependent on one another. Therefore, a recursive solution similar to that in Chapter~\ref{chap:noregen} is no longer possible and eq\ref{eq:regenrecurrence} should instead be treated as a linear system of $h$ equations. Furthermore, the time shift in the $\tau$-dependency splits them further making the system actually $hT_R/\gcd(T_R, T_A)$-dimensional.
For example in case of fighting ankous with a scimitar ($h=60$, $T_R=100$, $T_A=4$) we would have to solve a system of 1500 equations.
\pagebreak


\subsection{Random healing cycle approximation}
Instead of regenerating deterministically at realistic time intervals we could assume that the healing happens right after each hit with such a probability that the correct regeneration rate is achieved. This reduces the complexity of equation~\ref{eq:regenrecurrence} significantly as it removes the $\tau$-dependence entirely. We define the \emph{regeneration rate} as
\begin{align}\label{eq:regenProbability}
    \rho = \frac{T_A}{T_R}
\end{align}
and interpret it as the probability that a regeneration event occurs during an attack cycle.

In this model, there are two ways of lowering the hitpoints by $k$: hit $k$ and heal 0 or hit $k+1$ and heal 1. In terms of the regeneration probability $\rho$ and the damage roll $X$ the transition probabilities are
\begin{align}
    p_{ij}
         &= \begin{cases}
			 \Pr{X = j-i} + \rho \Pr{X = j-i+1} \quad &\mbox{if } i = h \\
            (1-\rho)\Pr{X = j-i} + \rho \Pr{X = j-i+1} \quad &\mbox{if } 1 < i < h \\
            (1-\rho)\Pr{X = j-i} \quad &\mbox{if } i = 1 \\
            \Pr{X \geq j-i} \quad &\mbox{if } i = 0
        \end{cases}\label{eq:damageDistributionRegen}.
\end{align}
Notice that transitioning to state 0 does not depend on regeneration because dead enemies cannot regenerate. For the same reason it is impossible to land in state 1 by first reaching 0 hitpoints and then healing. Transition probability for going to state $h$ on the other hand, is different because it is not possible to heal past maximum hitpoints.

Naturally it is still possible for the remaining hitpoints to climb up the state space making a recursive solution impossible. However, because of the eliminated dependence on the pahse of the regeneration cycle the walker is now fully described by equation~\ref{eq:fightLengthRecursion} just like in the regenerationless case. This reduces the size of the linear system down to $h$ equations, which is a much more manageable number for practical calculations.
\subsection{Effective hitpoints approach}
%Let $D_j^\tau$ be the total damage dealt in a fight. As in Chapter~\ref{chap:regen-walker} we let the indices $\tau$ and $j$ to denote the phase of the regeneration cycle and the remaining hitpoints. If we for now assume that the hitpoints are uncapped, then $D_j^\tau = j + N_j^\tau$, where $N_j^\tau$ is the number of regeneration attempts during the fight. If the fight has length $L_j^\tau$ it lasts for $T_A L_j^\tau$ ticks. The set of ticks on which the regeneration attempts during the fight occur is $\{\tau, \tau+T_R, \ldots, \tau + (N_j^\tau-1) T_R\}$

Consider a fight of length $L$. Since the fight lasts $T_A L$ ticks the number of hitpoints regenerated can be estimated by $\frac{T_A}{T_R} L$. Assuming that the enemy has $h$ maximum hitpoints, the total damage dealt during the fight is $y \equiv h + \frac{T_A}{T_R} L$. This quantity is called the \textit{effective hitpoints} because the fight is nearly equivalent to one against a non-regenerating enemy with $y$ hitpoints. If regeneration rate is small compared to the damage rate the expected length of a fight should be approximately $\langle L_y \rangle$. This gives us the equation
\begin{align}\label{eq:effHp}
	y = h + \rho\langle L_y \rangle
\end{align}
where we have defined the \textit{regeneration rate} $\rho \equiv \frac{T_A}{T_R}$.

To solve $y$ from this equation we use the asymptotic approximation (eq\ref{eq:asymptoticAppr}).
\begin{align}
	y &= h + \frac{2\rho}{ma} \left(y + \frac{m-1}{3}\right)\nonumber\\
	\implies \left(1 - \frac{2\rho}{ma}\right) y &= h + \frac{2\rho}{ma} \frac{m-1}{3}\nonumber\\
	\implies y
		&= \frac{1}{ma - 2\rho}\left(mah + 2\rho \frac{m-1}{3}\right)\nonumber\\
		&= h + \rho\left(\frac{h + \frac{m-1}{3}}{\frac{ma}{2} - \rho}\right)
\end{align}
Writing the effective hitpoints in this form reveals some nice properties. Comparing the result to eq\ref{eq:effHp} allows reading off the expected length of a fight as
\begin{align}
	\langle L_y \rangle = \frac{h + \frac{m-1}{3}}{\frac{ma}{2} - \rho}
\end{align}
and in turn the damage per hit as
\begin{align}
	\frac{y}{\langle L_y \rangle}
		&= \frac{\frac{ma}{2} - \rho}{1 + \frac{m-1}{3h}} + \rho
\end{align}
If the regeneration rate $\rho$ is larger than the damage rate $\frac{ma}{2}$ the equation breaks down as the term $\frac{ma}{2} - \rho$ becomes negative. This seems to suggest that in this case the fight would go on forever. Of course this is just a limitation of the model since in reality the fight would eventually terminate as long as $T_R > T_A$ and $m > 0$, which is a very resonable assumption.

