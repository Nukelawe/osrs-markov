Consider a scenario in which a single attacker is fighting an unlimited supply of identical passive enemies one at a time. We denote the \textbf{maximum hitpoints} of an enemy by $h$. Once the remaining hitpoints of an enemy reaches 0 the enemy dies and is immediately replaced by a new one with all of its hitpoints remaining. During a fight, the enemy is hit periodically, once every \textbf{attack period} (denoted by $T_A$) and the damage dealt is calulated as follows.
\begin{enumerate}
	\item The game determines if the hit will succeed by some random process, which depends on various parameters such as the defensive bonuses of the enemy, offensive bonuses of the attacker and appropriate combat-related stats. We ignore the details of this process and simply call the probability of success the \textbf{accuracy} (denoted by $a$). If the accuracy check fails, the attack is considered a miss and 0 damage is dealt.
	\item If the accuracy check is successful, damage roll $M\sim U[0,m]$, a uniform random integer between 0 and $m$, is chosen, where $m$ is the \textbf{maximum hit}. Again, we do not concern ourselves with the specifics of how the max hit is determined.
    \item If $M > H$, where $H$ is the remaining hitpoints of the enemy, the damage is capped to $H$ and the final damage dealt is $\min(M,H)$. This is the overkill effect.
\end{enumerate}
Notice that it is possible for the hit to deal no damage even if the accuracy check succeeds because the damage roll $M$ could be 0. For the specifics of max hit and accuracy calculation see for example the \href{https://oldschool.runescape.wiki/w/Maximum_hit}{OSRSwiki articles}~\cite{wiki} on maximum hit and the \href{https://github.com/Palfore/OSRSmath}{OSRS combat overview}~\cite{palfore} by Palfore.
