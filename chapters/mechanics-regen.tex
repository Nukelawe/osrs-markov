Natural regeneration is a process which periodically attempts to increment the remaining hitpoints by one. A regeneration attempt will fail if the hitpoints are already full. The period $T_R$ of the healing cycle is called the \textbf{regeneration period} and can vary between enemies giving rise to different regeneration rates. At the beginning of a fight the state of the healing cycle can be assumed to be unknown and thus, treated as a uniformly distributed random variable $t_0 \sim U[0,T_R-1]$. This way only the timing of the first regeneration attempt is random and the rest are perfectly periodic.

If measured in \href{https://oldschool.runescape.wiki/w/RuneScape_clock}{ticks}~\cite{wikitick}, the periods $T_A$ and $T_R$ as well as all other quantities that describe time can be assumed integers. We also assume $T_R > T_A$ which ensures that there will be at most one regeneration attempt between any two hits. This assumption should hold for nearly all cases in practice as the the typical regeneration period is 100 ticks (60 seconds) and even the slowest weapons have attack periods of just 7 ticks (4.2 seconds). Another assumption is that the damage dealt by a hit is calculated \emph{after} regeneration if a regeneration attempt occurs on the same tick as the hit. Doing so avoids the edge case in which the enemy dies and then immediately regenerates back to life.
